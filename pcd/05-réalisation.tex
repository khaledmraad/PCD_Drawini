\chapter{Réalisation}
\label{chap:realisation}

\section{Introduction}

Dans cette section, nous aborderons la mise en œuvre pratique de notre application de conversion de wireframes dessinées à la main en interfaces web. Nous détaillerons les environnements de travail utilisés, les choix techniques effectués, ainsi que la réalisation de chaque sprint.

\section{Environnement de travail}

\subsection{Environnement Logiciel}

Pour le développement de l'application, nous avons utilisé les outils suivants :

\begin{itemize}
    \item \textbf{IDE (Environnement de Développement Intégré) :} Nous avons choisi Visual Studio Code comme IDE principal pour son interface conviviale, sa prise en charge des langages de programmation utilisés (JavaScript, Java, Python, etc.) et sa large gamme d'extensions.
    \item \textbf{Système de Gestion de Versions :} Git a été utilisé pour le contrôle de version du code source, avec GitHub comme plateforme de stockage distant.
    \item \textbf{Outils de Collaboration :} Nous avons utilisé Slack pour la communication interne de l'équipe, ainsi que Trello pour la gestion des tâches et le suivi de l'avancement du projet.
\end{itemize}

\subsection{Choix techniques}

Pour le développement de chaque composant de l'application, nous avons fait les choix techniques suivants :

\begin{itemize}
    \item \textbf{Front-end :} Nous avons opté pour React.js en raison de sa modularité, de sa réactivité et de sa grande communauté de développeurs. Pour la conception de l'interface utilisateur, nous avons utilisé des bibliothèques et des frameworks comme Material-UI pour assurer une expérience utilisateur cohérente et moderne.
    \item \textbf{Back-end :} Spring Boot a été choisi pour le développement du back-end en raison de sa facilité de configuration, de son intégration transparente avec d'autres frameworks et bibliothèques Java, ainsi que de sa prise en charge de MongoDB, notre choix de base de données.
    \item \textbf{Base de données :} Nous avons utilisé MongoDB, une base de données NoSQL, pour sa flexibilité et sa capacité à gérer des données semi-structurées, ce qui était approprié pour le stockage des wireframes et des informations utilisateur.
    \item \textbf{Outils de conception :} Figma a été utilisé pour la création des wireframes et des maquettes de l'interface utilisateur en raison de sa facilité d'utilisation, de sa collaboration en temps réel et de sa capacité à exporter des ressources pour une utilisation ultérieure dans le développement.
\end{itemize}

\section{Réalisation des sprints}

\subsection{Sprint 1 : Front-end}

Ce sprint a été dédié à la mise en place de l'infrastructure front-end de l'application. Les tâches principales effectuées comprenaient :

\begin{itemize}
    \item Création de l'arborescence des composants React.js.
    \item Définition du thème et de la palette de couleurs de l'interface utilisateur.
    \item Mise en place de la navigation entre les différentes pages de l'application.
    \item Intégration des composants Material-UI pour assurer la cohérence visuelle.
\end{itemize}

\subsection{Sprint 2 : Authentification et Autorisation}

Dans ce sprint, l'équipe a travaillé sur la mise en place de l'authentification et de l'autorisation des utilisateurs. Les principales tâches effectuées étaient :

\begin{itemize}
    \item Conception et développement du système d'inscription (Sign up) avec validation des données et envoi de notifications par email.
    \item Implémentation du système de connexion (Login) avec gestion des sessions utilisateur.
    \item Configuration de Spring Security pour sécuriser les endpoints sensibles de l'API.
\end{itemize}

\subsection{Sprint 3 : Implémentation du modèle d'IA de Détection et de Classification}

Ce sprint a été axé sur l'intégration d'un modèle d'intelligence artificielle pour la détection et la classification des éléments dans les wireframes. Les principales tâches effectuées étaient :

\begin{itemize}
    \item Collecte et prétraitement des données pour l'entraînement du modèle.
    \item Choix de l'architecture du modèle et entraînement sur un ensemble de données diversifié.
    \item Intégration du modèle dans l'application et mise en place de l'API pour son utilisation.
\end{itemize}

\subsection{Sprint 4 : Conversion en Code}

Dans ce sprint final, l'équipe s'est concentrée sur la conversion des wireframes détectées en code HTML/CSS ou React.js. Les principales tâches effectuées étaient :

\begin{itemize}
    \item Développement du service backend avec Flask pour la conversion des données JSON en code web.
    \item Intégration du service dans l'application et validation de la génération de code.
    \item Tests unitaires et d'intégration pour assurer la fiabilité et la qualité du code généré.
\end{itemize}

\section{Conclusion}

Ce chapitre a présenté en détail la réalisation de l'application, en mettant en évidence les environnements de travail utilisés, les choix techniques effectués et la réalisation de chaque sprint. Ces informations fournissent un aperçu complet du processus de développement de l'application de conversion de wireframes en interfaces web.

